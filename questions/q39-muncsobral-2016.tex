{\bf Questão 39} 
 Uma função é dita injetiva quando elementos diferentes (no domínio) têm imagens diferentes (no contra-
dominio). Quando todo elemento do contra-domínio é imagem de algum elemento do domínio, diz-se que a função
é sobrejetiva. Assinale a opção correta:

\begin{enumerate}
		\item ( ) A função $f(x)= x^2$ com domínio e contra-dominio no conjunto  dos números reais, é injetiva.
		\item ( ) A função $f(x)= 2x+1$ com domínio  e contra-domínio  sendo o conjunto dos números reais não negativos, é sobrejetiva.
		\item ( ) A função $f(x)= 2x$ com domínio e contra-domínio no conjunto dos números reais, é injetiva e sobrejetiva.
		\item ( ) A função $f(x)= x^2 +1$ com domínio e contra-domínlo no conjunto dos números reais não negativos, é injetiva e sobrejetiva.
\end{enumerate}
