\SOL{
\begin{center}
\begin{tikzpicture}
		\tkzDefPoints{0/0/A, 6/0/B}
		%\tkzDrawTriangle[pythagore](A,B)\tkzGetPoint{C}
		\tkzDrawTriangle[two angles=36 and 72](A,B)\tkzGetPoint{C}
		%\tkzDrawTriangle[isosceles right](C,B)\tkzGetPoint{A}
		%\tkzDrawTriangle[euclid](A,B)\tkzGetPoint{C}
		\tkzDrawSegment[dim={ $x$,-0.5cm,}](A,B)
		\tkzDrawSegment[dim={ $y$,-0.5cm,}](B,C)
		\tkzMarkAngle[size=.7, fill=gray!20, opacity=1.3](B,A,C)
				\tkzLabelAngle[pos=-1.1](C,A,B){$36^\circ$}
				%\tkzText(2,2){$\dfrac{AC}{AB}=\varphi$}
		\tkzDefLine[bisector,normed](A,B,C) \tkzGetPoint{b}
		\tkzDrawLine[dashed,color=blue, add=0 and 4](B,b)
		\tkzInterLL(B,b)(A,C)\tkzGetPoint{D}
		\tkzShowLine[bisector,gap=4,size=2,color=red](A,B,C)

		\tkzDrawPoints(A,B,C,D)
		\tkzLabelPoints(B)
		\tkzLabelPoints[left](C)
		\tkzLabelPoints[above](A,D)
\end{tikzpicture}
\end{center}

Como o ângulo em $A$ é único então a base do triângulo é o lado $BC$ e $AC=AB$, 
com os ângulos da base medindo $2\widehat B+36^\circ=180^\circ$, ou seja, $\widehat C=\widehat B=\frac{180-36}{2}=72^\circ$.


\begin{center}
\begin{tikzpicture}
		\tkzDefPoints{0/0/A, 6/0/B}
		\tkzDrawTriangle[two angles=36 and 72](A,B)\tkzGetPoint{C}
		\tkzDrawSegment[dim={ $x$,-0.5cm,}](A,B)
		\tkzDrawSegment[dim={ $x$,0.5cm,}](A,C)
		\tkzDrawSegment[dim={ $y$,-0.5cm,}](B,C)
		\tkzMarkAngle[size=.7, fill=gray!20, opacity=1.3](B,A,C)
		\tkzMarkAngle[size=.5, fill=gray!20, opacity=1.3](C,B,A)
		\tkzMarkAngle[size=.5, fill=gray!20, opacity=1.3](A,C,B)
				\tkzLabelAngle[pos=-1.1](C,A,B){$36^\circ$}
				\tkzLabelAngle[pos=0.9](A,C,B){$72^\circ$}
				\tkzLabelAngle[pos=0.9](C,B,A){$72^\circ$}
		\tkzDefLine[bisector,normed](A,B,C) \tkzGetPoint{b}
		\tkzDrawLine[dashed,color=blue, add=0 and 3](B,b)
		\tkzInterLL(B,b)(A,C)\tkzGetPoint{D}
		%\tkzShowLine[bisector,gap=4,size=2,color=red](A,B,C)

		\tkzDrawPoints(A,B,C,D)
		\tkzLabelPoints(B)
		\tkzLabelPoints[left](C)
		\tkzLabelPoints[above](A,D)
\end{tikzpicture}
\end{center}


Como $BD$ é bissetriz de $A\widehat B C$, pelo teorema da {\sc bissetriz interna}: %o ângulo $A\widehat B D=\frac{72^\circ}{2}=36^\circ$

\begin{eqnarray*}
		\frac{\overline{AB}}{\overline{AD}} &= & \frac{\overline{CB}}{\overline{CD}}\\
		\frac{x}{\overline{AD}} &= & \frac{y}{x-\overline{AD}}\\
		x(x-\overline{AD}) &= & y\overline{AD}\\
		x^2 &= & x\overline{AD} + y\overline{AD}\\
		x^2 &= & (x+y)\overline{AD} \\
\end{eqnarray*}
}
\todo{Infelizmente meu notebook tá com problema e as figuras não compilaram. Vou corrigir no próximo update}
