\SOL{
Temos que 
\begin{eqnarray*}
		\begin{cases}
		p(x)=(x-1)q_1(x)+5\\
		p(x)=(x+1)q_2(x)+5
		\end{cases}
\end{eqnarray*}
Usando o Teorema do Resto:
\begin{eqnarray*}
		p(1)=5\\
		p(-1)=5\\
\end{eqnarray*}

Queremos determinar $R(x)$ na equação:
\begin{eqnarray*}
		p(x)=(x-1)(x+1)q(x)+R(x)
\end{eqnarray*}
Como $(x-1)(x+1)$ tem grau $2$, logo o resto $R(x)$ tem no máximo grau $1$ ,ou sejam $R(x)=ax+b$, reescrevendo:
\begin{eqnarray*}
		p(x)=(x-1)(x+1)q(x)+ax+b
\end{eqnarray*}
Aplicando os valores $p(1)$ e $p(-1)$:
\begin{eqnarray*}
		\begin{cases}
		p(1)=(1-1)(1+1)q(1)+a\cdot 1+b=a+b=5\\
		p(-1)=(-1-1)(-1+1)q(-1)+a\cdot (-1)+b=-a+b=5
		\end{cases}
\end{eqnarray*}
\begin{eqnarray*}
		\begin{cases}
		a+b=5\\
		-a+b=5
		\end{cases}
\end{eqnarray*}

Resolvendo o sistema, encontramos: $a=0$ e $b=5$. Portando, $R(x)=5$.

}
\todo{O teorema do Resto de Polinômios {\sc Afirma} que divisão de um polinômio $p(x)$ pelo binômio $ax+b$ tem como resto $p(\frac{-b}{a}$)}
