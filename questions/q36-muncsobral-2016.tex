{\bf Questão 36} 
 Para comprar camisas e calções, um homem dispõe de uma certa quantia. Na loja $A$ o calção custa $R\$ 40,00$ e
a camisa custa $R\$ 60,00$. Já na loja $B$, o calção custa $R\$ 35,00$ e a camisa custa $R\$ 70,00$. Independentemente da
escolha da loja, o número de calções comprados não mudará. O mesmo vale para o número de camisas. Nestas
condições, para que o valor total da compra seja o mesmo em ambas as lojas:

\begin{enumerate}
		\item ( ) a quantidade de calções deve ser igual à quantidade de camisas.
		\item ( ) a quantidade de calções deve ser o dobro da quantidade de camisas.
		\item ( ) a quantidade de calções deve ser a metade da quantidade de camisas.
		\item ( ) a quantidade de calções deve ser um terço da quantidade de camisas.
\end{enumerate}
